Power rule can be considered as the one of the most fundamental rules in calculus.
Most of us remember this law from the first calculus course
\begin{align*}
    \odv{x^n}{x} = nx^{n-1}
\end{align*}
where $n$ is a constant.
One of the common strategies to prove the power rule is by utilizing limit definition
of derivative in conjunction with binomial theorem.
Recall the limit form of derivative
\begin{align*}
    \odv{f(x)}{x} = \lim_{h \to 0} \left[ \frac{f(x+h) - f(x)}{h} \right]
\end{align*}
where $f(x)$ is defined over $\mathbb{R}$ and at least of smoothness class $C^1$.
Let be $f(x) = x^n$ with constant $n$.
Then its derivative is
\begin{align*}
    \odv{x^n}{x} = \lim_{h \to 0} \left[ \frac{(x+h)^n - x^n}{h} \right]
\end{align*}
Notice that we can express the function's growth $(x+h)^n$ by using binomial theorem
\begin{align*}
(x+h)
    ^n = \sum_{k=0}^{n} \binom{n}{k} x^{n-k} h^k
\end{align*}
So that
\begin{align*}
    \odv{x^n}{x} &= \lim_{h \to 0} \left[ \frac{1}{h} \sum_{k=1}^{n} \binom{n}{k} x^{n-k} h^k \right] \\
    &= \lim_{h \to 0} \left[ \binom{n}{1} x^{n-1} +  \binom{n}{2} x^{n-2} h
    + \binom{n}{3} x^{n-3} h^2 + \dots + \binom{n}{n} x^{0} h^n \right] \\
    &= \binom{n}{1} x^{n-1}
\end{align*}
However, is the binomial theorem the only polynomial identity to express the growth rate?
Well, not really, we can utilize different approach to express polynomial growth.
Consider the following identity for odd powers~\cite{kolosov2024history, kolosov2016link, kolosov2022106, kolosov2023polynomial}
\begin{equation}
(x-2a)
    ^{2m+1} = \sum_{r=0}^{m} \coeffA{m}{r} \sum_{k=a+1}^{x-a} (k-a)^r (x-k-a)^r
    \label{eq:odd-power-identity}
\end{equation}
where $\coeffA{m}{r}$ is a real coefficient defined recursively
\begin{equation}
    \label{eq:definition_coefficient_a}
    \coeffA{m}{r} =
    \begin{cases}
    (2r+1)
        \binom{2r}{r} & \mathrm{if} \; r=m \\
        (2r+1) \binom{2r}{r} \sum_{d \geq 2r+1}^{m} \coeffA{m}{d} \binom{d}{2r+1} \frac{(-1)^{d-1}}{d-r}
        \bernoulli{2d-2r} & \mathrm{if} \; 0 \leq r<m \\
        0 & \mathrm{if} \; r<0 \; \mathrm{or} \; r>m
    \end{cases}
\end{equation}
where $\bernoulli{t}$ are Bernoulli numbers~\cite{bateman1953higher} such that $\bernoulli{1}=\frac{1}{2}$.
For example,
\begin{table}[H]
    \begin{center}
        \setlength\extrarowheight{-6pt}
        \begin{tabular}{c|cccccccc}
            $m/r$ & 0 & 1       & 2      & 3      & 4   & 5    & 6     & 7 \\ [3px]
            \hline
            0     & 1 &         &        &        &     &      &       &       \\
            1     & 1 & 6       &        &        &     &      &       &       \\
            2     & 1 & 0       & 30     &        &     &      &       &       \\
            3     & 1 & -14     & 0      & 140    &     &      &       &       \\
            4     & 1 & -120    & 0      & 0      & 630 &      &       &       \\
            5     & 1 & -1386   & 660    & 0      & 0   & 2772 &       &       \\
            6     & 1 & -21840  & 18018  & 0      & 0   & 0    & 12012 &       \\
            7     & 1 & -450054 & 491400 & -60060 & 0   & 0    & 0     & 51480
        \end{tabular}
    \end{center}
    \caption{Coefficients $\coeffA{m}{r}$. See OEIS sequences~\cite{kolosov2018numerator,kolosov2018denominator}.}
    \label{tab:table_of_coefficients_a}
\end{table}

Properties of coefficients $\coeffA{m}{r}$
\begin{itemize}
    \item $\coeffA{m}{m} = \binom{2m}{m}$
    \item $\coeffA{m}{r} = 0$ for $m < 0$ and $r > m$
    \item $\coeffA{m}{r} = 0$ for $r < 0$
    \item $\coeffA{m}{r} = 0$ for $\frac{m}{2} \leq r < m$
    \item $\coeffA{m}{0} = 1$ for $m \geq 0$
    \item $\coeffA{m}{r}$ are integers for $m \leq 11$
    \item Row sums: $\sum_{r=0}^{m} \coeffA{m}{r} = 2^{2m+1} - 1$
\end{itemize}
Therefore, by setting $a=-\frac{h}{2}$ in~\eqref{eq:odd-power-identity} we can express the growth rate of odd powers as
\begin{align*}
(x+h)
    ^{2m+1}
    = \sum_{r=0}^{m} \coeffA{m}{r} \sum_{k=-\frac{h}{2}+1}^{x+\frac{h}{2}} \left( k+\frac{h}{2} \right)^{r} \left(x-k+\frac{h}{2} \right)^{r}
    = \sum_{r=0}^{m} \coeffA{m}{r} \sum_{k=1}^{x+h} k^r \left(x-k+h \right)^{r}
\end{align*}
Thus,
\begin{align*}
    \odv{x^{2m+1}}{x}
    &= \lim_{h \to 0} \frac{1}{h} \left[ \sum_{r=0}^{m} \coeffA{m}{r} \sum_{k=-\frac{h}{2}+1}^{x+\frac{h}{2}} \left( k+\frac{h}{2} \right)^{r} \left(x-k+\frac{h}{2} \right)^{r}- \sum_{r=0}^{m} \coeffA{m}{r} \sum_{k=1}^{x} k^r (x-k)^r\right] \\
    &= \lim_{h \to 0} \frac{1}{h} \left[ \sum_{r=0}^{m} \coeffA{m}{r} \left( \sum_{k=-\frac{h}{2}+1}^{x+\frac{h}{2}} \left( k+\frac{h}{2} \right)^{r} \left(x-k+\frac{h}{2} \right)^{r}- \sum_{k=1}^{x} k^r (x-k)^r \right) \right] \\
    &= \lim_{h \to 0} \frac{1}{h} \left[ \sum_{r=0}^{m} \coeffA{m}{r} \left( \sum_{k=1}^{x+h} k^r \left(x-k+h \right)^{r} - \sum_{k=1}^{x} k^r \left(x-k \right)^{r} \right) \right]
\end{align*}
For instance, having $m=1,2$
\begin{align*}
    &\sum_{r=0}^{1} \coeffA{1}{r} \sum_{k=-\frac{h}{2}+1}^{x+\frac{h}{2}} \left( k+\frac{h}{2} \right)^{r} \left(x-k+\frac{h}{2} \right)^{r}
    = h + x + (-1 + h + x) (h + x) (1 + h + x) \\
    &\sum_{r=0}^{2} \coeffA{2}{r} \sum_{k=-\frac{h}{2}+1}^{x+\frac{h}{2}} \left( k+\frac{h}{2} \right)^{r} \left(x-k+\frac{h}{2} \right)^{r} \\
    &= h + x + (h + x) (1 + h + x) (-1 + h - h^2 + h^3 + x - 2 h x + 3 h^2 x - x^2 + 3 h x^2 + x^3)
\end{align*}
Which further simplifies to binomial theorem.
